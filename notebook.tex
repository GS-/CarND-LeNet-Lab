
% Default to the notebook output style

    


% Inherit from the specified cell style.




    
\documentclass[11pt]{article}

    
    
    \usepackage[T1]{fontenc}
    % Nicer default font (+ math font) than Computer Modern for most use cases
    \usepackage{mathpazo}

    % Basic figure setup, for now with no caption control since it's done
    % automatically by Pandoc (which extracts ![](path) syntax from Markdown).
    \usepackage{graphicx}
    % We will generate all images so they have a width \maxwidth. This means
    % that they will get their normal width if they fit onto the page, but
    % are scaled down if they would overflow the margins.
    \makeatletter
    \def\maxwidth{\ifdim\Gin@nat@width>\linewidth\linewidth
    \else\Gin@nat@width\fi}
    \makeatother
    \let\Oldincludegraphics\includegraphics
    % Set max figure width to be 80% of text width, for now hardcoded.
    \renewcommand{\includegraphics}[1]{\Oldincludegraphics[width=.8\maxwidth]{#1}}
    % Ensure that by default, figures have no caption (until we provide a
    % proper Figure object with a Caption API and a way to capture that
    % in the conversion process - todo).
    \usepackage{caption}
    \DeclareCaptionLabelFormat{nolabel}{}
    \captionsetup{labelformat=nolabel}

    \usepackage{adjustbox} % Used to constrain images to a maximum size 
    \usepackage{xcolor} % Allow colors to be defined
    \usepackage{enumerate} % Needed for markdown enumerations to work
    \usepackage{geometry} % Used to adjust the document margins
    \usepackage{amsmath} % Equations
    \usepackage{amssymb} % Equations
    \usepackage{textcomp} % defines textquotesingle
    % Hack from http://tex.stackexchange.com/a/47451/13684:
    \AtBeginDocument{%
        \def\PYZsq{\textquotesingle}% Upright quotes in Pygmentized code
    }
    \usepackage{upquote} % Upright quotes for verbatim code
    \usepackage{eurosym} % defines \euro
    \usepackage[mathletters]{ucs} % Extended unicode (utf-8) support
    \usepackage[utf8x]{inputenc} % Allow utf-8 characters in the tex document
    \usepackage{fancyvrb} % verbatim replacement that allows latex
    \usepackage{grffile} % extends the file name processing of package graphics 
                         % to support a larger range 
    % The hyperref package gives us a pdf with properly built
    % internal navigation ('pdf bookmarks' for the table of contents,
    % internal cross-reference links, web links for URLs, etc.)
    \usepackage{hyperref}
    \usepackage{longtable} % longtable support required by pandoc >1.10
    \usepackage{booktabs}  % table support for pandoc > 1.12.2
    \usepackage[inline]{enumitem} % IRkernel/repr support (it uses the enumerate* environment)
    \usepackage[normalem]{ulem} % ulem is needed to support strikethroughs (\sout)
                                % normalem makes italics be italics, not underlines
    

    
    
    % Colors for the hyperref package
    \definecolor{urlcolor}{rgb}{0,.145,.698}
    \definecolor{linkcolor}{rgb}{.71,0.21,0.01}
    \definecolor{citecolor}{rgb}{.12,.54,.11}

    % ANSI colors
    \definecolor{ansi-black}{HTML}{3E424D}
    \definecolor{ansi-black-intense}{HTML}{282C36}
    \definecolor{ansi-red}{HTML}{E75C58}
    \definecolor{ansi-red-intense}{HTML}{B22B31}
    \definecolor{ansi-green}{HTML}{00A250}
    \definecolor{ansi-green-intense}{HTML}{007427}
    \definecolor{ansi-yellow}{HTML}{DDB62B}
    \definecolor{ansi-yellow-intense}{HTML}{B27D12}
    \definecolor{ansi-blue}{HTML}{208FFB}
    \definecolor{ansi-blue-intense}{HTML}{0065CA}
    \definecolor{ansi-magenta}{HTML}{D160C4}
    \definecolor{ansi-magenta-intense}{HTML}{A03196}
    \definecolor{ansi-cyan}{HTML}{60C6C8}
    \definecolor{ansi-cyan-intense}{HTML}{258F8F}
    \definecolor{ansi-white}{HTML}{C5C1B4}
    \definecolor{ansi-white-intense}{HTML}{A1A6B2}

    % commands and environments needed by pandoc snippets
    % extracted from the output of `pandoc -s`
    \providecommand{\tightlist}{%
      \setlength{\itemsep}{0pt}\setlength{\parskip}{0pt}}
    \DefineVerbatimEnvironment{Highlighting}{Verbatim}{commandchars=\\\{\}}
    % Add ',fontsize=\small' for more characters per line
    \newenvironment{Shaded}{}{}
    \newcommand{\KeywordTok}[1]{\textcolor[rgb]{0.00,0.44,0.13}{\textbf{{#1}}}}
    \newcommand{\DataTypeTok}[1]{\textcolor[rgb]{0.56,0.13,0.00}{{#1}}}
    \newcommand{\DecValTok}[1]{\textcolor[rgb]{0.25,0.63,0.44}{{#1}}}
    \newcommand{\BaseNTok}[1]{\textcolor[rgb]{0.25,0.63,0.44}{{#1}}}
    \newcommand{\FloatTok}[1]{\textcolor[rgb]{0.25,0.63,0.44}{{#1}}}
    \newcommand{\CharTok}[1]{\textcolor[rgb]{0.25,0.44,0.63}{{#1}}}
    \newcommand{\StringTok}[1]{\textcolor[rgb]{0.25,0.44,0.63}{{#1}}}
    \newcommand{\CommentTok}[1]{\textcolor[rgb]{0.38,0.63,0.69}{\textit{{#1}}}}
    \newcommand{\OtherTok}[1]{\textcolor[rgb]{0.00,0.44,0.13}{{#1}}}
    \newcommand{\AlertTok}[1]{\textcolor[rgb]{1.00,0.00,0.00}{\textbf{{#1}}}}
    \newcommand{\FunctionTok}[1]{\textcolor[rgb]{0.02,0.16,0.49}{{#1}}}
    \newcommand{\RegionMarkerTok}[1]{{#1}}
    \newcommand{\ErrorTok}[1]{\textcolor[rgb]{1.00,0.00,0.00}{\textbf{{#1}}}}
    \newcommand{\NormalTok}[1]{{#1}}
    
    % Additional commands for more recent versions of Pandoc
    \newcommand{\ConstantTok}[1]{\textcolor[rgb]{0.53,0.00,0.00}{{#1}}}
    \newcommand{\SpecialCharTok}[1]{\textcolor[rgb]{0.25,0.44,0.63}{{#1}}}
    \newcommand{\VerbatimStringTok}[1]{\textcolor[rgb]{0.25,0.44,0.63}{{#1}}}
    \newcommand{\SpecialStringTok}[1]{\textcolor[rgb]{0.73,0.40,0.53}{{#1}}}
    \newcommand{\ImportTok}[1]{{#1}}
    \newcommand{\DocumentationTok}[1]{\textcolor[rgb]{0.73,0.13,0.13}{\textit{{#1}}}}
    \newcommand{\AnnotationTok}[1]{\textcolor[rgb]{0.38,0.63,0.69}{\textbf{\textit{{#1}}}}}
    \newcommand{\CommentVarTok}[1]{\textcolor[rgb]{0.38,0.63,0.69}{\textbf{\textit{{#1}}}}}
    \newcommand{\VariableTok}[1]{\textcolor[rgb]{0.10,0.09,0.49}{{#1}}}
    \newcommand{\ControlFlowTok}[1]{\textcolor[rgb]{0.00,0.44,0.13}{\textbf{{#1}}}}
    \newcommand{\OperatorTok}[1]{\textcolor[rgb]{0.40,0.40,0.40}{{#1}}}
    \newcommand{\BuiltInTok}[1]{{#1}}
    \newcommand{\ExtensionTok}[1]{{#1}}
    \newcommand{\PreprocessorTok}[1]{\textcolor[rgb]{0.74,0.48,0.00}{{#1}}}
    \newcommand{\AttributeTok}[1]{\textcolor[rgb]{0.49,0.56,0.16}{{#1}}}
    \newcommand{\InformationTok}[1]{\textcolor[rgb]{0.38,0.63,0.69}{\textbf{\textit{{#1}}}}}
    \newcommand{\WarningTok}[1]{\textcolor[rgb]{0.38,0.63,0.69}{\textbf{\textit{{#1}}}}}
    
    
    % Define a nice break command that doesn't care if a line doesn't already
    % exist.
    \def\br{\hspace*{\fill} \\* }
    % Math Jax compatability definitions
    \def\gt{>}
    \def\lt{<}
    % Document parameters
    \title{LeNet-Lab-Solution}
    
    
    

    % Pygments definitions
    
\makeatletter
\def\PY@reset{\let\PY@it=\relax \let\PY@bf=\relax%
    \let\PY@ul=\relax \let\PY@tc=\relax%
    \let\PY@bc=\relax \let\PY@ff=\relax}
\def\PY@tok#1{\csname PY@tok@#1\endcsname}
\def\PY@toks#1+{\ifx\relax#1\empty\else%
    \PY@tok{#1}\expandafter\PY@toks\fi}
\def\PY@do#1{\PY@bc{\PY@tc{\PY@ul{%
    \PY@it{\PY@bf{\PY@ff{#1}}}}}}}
\def\PY#1#2{\PY@reset\PY@toks#1+\relax+\PY@do{#2}}

\expandafter\def\csname PY@tok@w\endcsname{\def\PY@tc##1{\textcolor[rgb]{0.73,0.73,0.73}{##1}}}
\expandafter\def\csname PY@tok@c\endcsname{\let\PY@it=\textit\def\PY@tc##1{\textcolor[rgb]{0.25,0.50,0.50}{##1}}}
\expandafter\def\csname PY@tok@cp\endcsname{\def\PY@tc##1{\textcolor[rgb]{0.74,0.48,0.00}{##1}}}
\expandafter\def\csname PY@tok@k\endcsname{\let\PY@bf=\textbf\def\PY@tc##1{\textcolor[rgb]{0.00,0.50,0.00}{##1}}}
\expandafter\def\csname PY@tok@kp\endcsname{\def\PY@tc##1{\textcolor[rgb]{0.00,0.50,0.00}{##1}}}
\expandafter\def\csname PY@tok@kt\endcsname{\def\PY@tc##1{\textcolor[rgb]{0.69,0.00,0.25}{##1}}}
\expandafter\def\csname PY@tok@o\endcsname{\def\PY@tc##1{\textcolor[rgb]{0.40,0.40,0.40}{##1}}}
\expandafter\def\csname PY@tok@ow\endcsname{\let\PY@bf=\textbf\def\PY@tc##1{\textcolor[rgb]{0.67,0.13,1.00}{##1}}}
\expandafter\def\csname PY@tok@nb\endcsname{\def\PY@tc##1{\textcolor[rgb]{0.00,0.50,0.00}{##1}}}
\expandafter\def\csname PY@tok@nf\endcsname{\def\PY@tc##1{\textcolor[rgb]{0.00,0.00,1.00}{##1}}}
\expandafter\def\csname PY@tok@nc\endcsname{\let\PY@bf=\textbf\def\PY@tc##1{\textcolor[rgb]{0.00,0.00,1.00}{##1}}}
\expandafter\def\csname PY@tok@nn\endcsname{\let\PY@bf=\textbf\def\PY@tc##1{\textcolor[rgb]{0.00,0.00,1.00}{##1}}}
\expandafter\def\csname PY@tok@ne\endcsname{\let\PY@bf=\textbf\def\PY@tc##1{\textcolor[rgb]{0.82,0.25,0.23}{##1}}}
\expandafter\def\csname PY@tok@nv\endcsname{\def\PY@tc##1{\textcolor[rgb]{0.10,0.09,0.49}{##1}}}
\expandafter\def\csname PY@tok@no\endcsname{\def\PY@tc##1{\textcolor[rgb]{0.53,0.00,0.00}{##1}}}
\expandafter\def\csname PY@tok@nl\endcsname{\def\PY@tc##1{\textcolor[rgb]{0.63,0.63,0.00}{##1}}}
\expandafter\def\csname PY@tok@ni\endcsname{\let\PY@bf=\textbf\def\PY@tc##1{\textcolor[rgb]{0.60,0.60,0.60}{##1}}}
\expandafter\def\csname PY@tok@na\endcsname{\def\PY@tc##1{\textcolor[rgb]{0.49,0.56,0.16}{##1}}}
\expandafter\def\csname PY@tok@nt\endcsname{\let\PY@bf=\textbf\def\PY@tc##1{\textcolor[rgb]{0.00,0.50,0.00}{##1}}}
\expandafter\def\csname PY@tok@nd\endcsname{\def\PY@tc##1{\textcolor[rgb]{0.67,0.13,1.00}{##1}}}
\expandafter\def\csname PY@tok@s\endcsname{\def\PY@tc##1{\textcolor[rgb]{0.73,0.13,0.13}{##1}}}
\expandafter\def\csname PY@tok@sd\endcsname{\let\PY@it=\textit\def\PY@tc##1{\textcolor[rgb]{0.73,0.13,0.13}{##1}}}
\expandafter\def\csname PY@tok@si\endcsname{\let\PY@bf=\textbf\def\PY@tc##1{\textcolor[rgb]{0.73,0.40,0.53}{##1}}}
\expandafter\def\csname PY@tok@se\endcsname{\let\PY@bf=\textbf\def\PY@tc##1{\textcolor[rgb]{0.73,0.40,0.13}{##1}}}
\expandafter\def\csname PY@tok@sr\endcsname{\def\PY@tc##1{\textcolor[rgb]{0.73,0.40,0.53}{##1}}}
\expandafter\def\csname PY@tok@ss\endcsname{\def\PY@tc##1{\textcolor[rgb]{0.10,0.09,0.49}{##1}}}
\expandafter\def\csname PY@tok@sx\endcsname{\def\PY@tc##1{\textcolor[rgb]{0.00,0.50,0.00}{##1}}}
\expandafter\def\csname PY@tok@m\endcsname{\def\PY@tc##1{\textcolor[rgb]{0.40,0.40,0.40}{##1}}}
\expandafter\def\csname PY@tok@gh\endcsname{\let\PY@bf=\textbf\def\PY@tc##1{\textcolor[rgb]{0.00,0.00,0.50}{##1}}}
\expandafter\def\csname PY@tok@gu\endcsname{\let\PY@bf=\textbf\def\PY@tc##1{\textcolor[rgb]{0.50,0.00,0.50}{##1}}}
\expandafter\def\csname PY@tok@gd\endcsname{\def\PY@tc##1{\textcolor[rgb]{0.63,0.00,0.00}{##1}}}
\expandafter\def\csname PY@tok@gi\endcsname{\def\PY@tc##1{\textcolor[rgb]{0.00,0.63,0.00}{##1}}}
\expandafter\def\csname PY@tok@gr\endcsname{\def\PY@tc##1{\textcolor[rgb]{1.00,0.00,0.00}{##1}}}
\expandafter\def\csname PY@tok@ge\endcsname{\let\PY@it=\textit}
\expandafter\def\csname PY@tok@gs\endcsname{\let\PY@bf=\textbf}
\expandafter\def\csname PY@tok@gp\endcsname{\let\PY@bf=\textbf\def\PY@tc##1{\textcolor[rgb]{0.00,0.00,0.50}{##1}}}
\expandafter\def\csname PY@tok@go\endcsname{\def\PY@tc##1{\textcolor[rgb]{0.53,0.53,0.53}{##1}}}
\expandafter\def\csname PY@tok@gt\endcsname{\def\PY@tc##1{\textcolor[rgb]{0.00,0.27,0.87}{##1}}}
\expandafter\def\csname PY@tok@err\endcsname{\def\PY@bc##1{\setlength{\fboxsep}{0pt}\fcolorbox[rgb]{1.00,0.00,0.00}{1,1,1}{\strut ##1}}}
\expandafter\def\csname PY@tok@kc\endcsname{\let\PY@bf=\textbf\def\PY@tc##1{\textcolor[rgb]{0.00,0.50,0.00}{##1}}}
\expandafter\def\csname PY@tok@kd\endcsname{\let\PY@bf=\textbf\def\PY@tc##1{\textcolor[rgb]{0.00,0.50,0.00}{##1}}}
\expandafter\def\csname PY@tok@kn\endcsname{\let\PY@bf=\textbf\def\PY@tc##1{\textcolor[rgb]{0.00,0.50,0.00}{##1}}}
\expandafter\def\csname PY@tok@kr\endcsname{\let\PY@bf=\textbf\def\PY@tc##1{\textcolor[rgb]{0.00,0.50,0.00}{##1}}}
\expandafter\def\csname PY@tok@bp\endcsname{\def\PY@tc##1{\textcolor[rgb]{0.00,0.50,0.00}{##1}}}
\expandafter\def\csname PY@tok@fm\endcsname{\def\PY@tc##1{\textcolor[rgb]{0.00,0.00,1.00}{##1}}}
\expandafter\def\csname PY@tok@vc\endcsname{\def\PY@tc##1{\textcolor[rgb]{0.10,0.09,0.49}{##1}}}
\expandafter\def\csname PY@tok@vg\endcsname{\def\PY@tc##1{\textcolor[rgb]{0.10,0.09,0.49}{##1}}}
\expandafter\def\csname PY@tok@vi\endcsname{\def\PY@tc##1{\textcolor[rgb]{0.10,0.09,0.49}{##1}}}
\expandafter\def\csname PY@tok@vm\endcsname{\def\PY@tc##1{\textcolor[rgb]{0.10,0.09,0.49}{##1}}}
\expandafter\def\csname PY@tok@sa\endcsname{\def\PY@tc##1{\textcolor[rgb]{0.73,0.13,0.13}{##1}}}
\expandafter\def\csname PY@tok@sb\endcsname{\def\PY@tc##1{\textcolor[rgb]{0.73,0.13,0.13}{##1}}}
\expandafter\def\csname PY@tok@sc\endcsname{\def\PY@tc##1{\textcolor[rgb]{0.73,0.13,0.13}{##1}}}
\expandafter\def\csname PY@tok@dl\endcsname{\def\PY@tc##1{\textcolor[rgb]{0.73,0.13,0.13}{##1}}}
\expandafter\def\csname PY@tok@s2\endcsname{\def\PY@tc##1{\textcolor[rgb]{0.73,0.13,0.13}{##1}}}
\expandafter\def\csname PY@tok@sh\endcsname{\def\PY@tc##1{\textcolor[rgb]{0.73,0.13,0.13}{##1}}}
\expandafter\def\csname PY@tok@s1\endcsname{\def\PY@tc##1{\textcolor[rgb]{0.73,0.13,0.13}{##1}}}
\expandafter\def\csname PY@tok@mb\endcsname{\def\PY@tc##1{\textcolor[rgb]{0.40,0.40,0.40}{##1}}}
\expandafter\def\csname PY@tok@mf\endcsname{\def\PY@tc##1{\textcolor[rgb]{0.40,0.40,0.40}{##1}}}
\expandafter\def\csname PY@tok@mh\endcsname{\def\PY@tc##1{\textcolor[rgb]{0.40,0.40,0.40}{##1}}}
\expandafter\def\csname PY@tok@mi\endcsname{\def\PY@tc##1{\textcolor[rgb]{0.40,0.40,0.40}{##1}}}
\expandafter\def\csname PY@tok@il\endcsname{\def\PY@tc##1{\textcolor[rgb]{0.40,0.40,0.40}{##1}}}
\expandafter\def\csname PY@tok@mo\endcsname{\def\PY@tc##1{\textcolor[rgb]{0.40,0.40,0.40}{##1}}}
\expandafter\def\csname PY@tok@ch\endcsname{\let\PY@it=\textit\def\PY@tc##1{\textcolor[rgb]{0.25,0.50,0.50}{##1}}}
\expandafter\def\csname PY@tok@cm\endcsname{\let\PY@it=\textit\def\PY@tc##1{\textcolor[rgb]{0.25,0.50,0.50}{##1}}}
\expandafter\def\csname PY@tok@cpf\endcsname{\let\PY@it=\textit\def\PY@tc##1{\textcolor[rgb]{0.25,0.50,0.50}{##1}}}
\expandafter\def\csname PY@tok@c1\endcsname{\let\PY@it=\textit\def\PY@tc##1{\textcolor[rgb]{0.25,0.50,0.50}{##1}}}
\expandafter\def\csname PY@tok@cs\endcsname{\let\PY@it=\textit\def\PY@tc##1{\textcolor[rgb]{0.25,0.50,0.50}{##1}}}

\def\PYZbs{\char`\\}
\def\PYZus{\char`\_}
\def\PYZob{\char`\{}
\def\PYZcb{\char`\}}
\def\PYZca{\char`\^}
\def\PYZam{\char`\&}
\def\PYZlt{\char`\<}
\def\PYZgt{\char`\>}
\def\PYZsh{\char`\#}
\def\PYZpc{\char`\%}
\def\PYZdl{\char`\$}
\def\PYZhy{\char`\-}
\def\PYZsq{\char`\'}
\def\PYZdq{\char`\"}
\def\PYZti{\char`\~}
% for compatibility with earlier versions
\def\PYZat{@}
\def\PYZlb{[}
\def\PYZrb{]}
\makeatother


    % Exact colors from NB
    \definecolor{incolor}{rgb}{0.0, 0.0, 0.5}
    \definecolor{outcolor}{rgb}{0.545, 0.0, 0.0}



    
    % Prevent overflowing lines due to hard-to-break entities
    \sloppy 
    % Setup hyperref package
    \hypersetup{
      breaklinks=true,  % so long urls are correctly broken across lines
      colorlinks=true,
      urlcolor=urlcolor,
      linkcolor=linkcolor,
      citecolor=citecolor,
      }
    % Slightly bigger margins than the latex defaults
    
    \geometry{verbose,tmargin=1in,bmargin=1in,lmargin=1in,rmargin=1in}
    
    

    \begin{document}
    
    
    \maketitle
    
    

    
    \hypertarget{lenet-lab-solution}{%
\section{LeNet Lab Solution}\label{lenet-lab-solution}}

\includegraphics{lenet.png} Source: Yan LeCun

    \hypertarget{load-data}{%
\subsection{Load Data}\label{load-data}}

Load the MNIST data, which comes pre-loaded with TensorFlow.

You do not need to modify this section.

    \begin{Verbatim}[commandchars=\\\{\}]
{\color{incolor}In [{\color{incolor}1}]:} \PY{k+kn}{from} \PY{n+nn}{tensorflow}\PY{n+nn}{.}\PY{n+nn}{examples}\PY{n+nn}{.}\PY{n+nn}{tutorials}\PY{n+nn}{.}\PY{n+nn}{mnist} \PY{k}{import} \PY{n}{input\PYZus{}data}
        
        \PY{n}{mnist} \PY{o}{=} \PY{n}{input\PYZus{}data}\PY{o}{.}\PY{n}{read\PYZus{}data\PYZus{}sets}\PY{p}{(}\PY{l+s+s2}{\PYZdq{}}\PY{l+s+s2}{MNIST\PYZus{}data/}\PY{l+s+s2}{\PYZdq{}}\PY{p}{,} \PY{n}{reshape}\PY{o}{=}\PY{k+kc}{False}\PY{p}{)}
        \PY{n}{X\PYZus{}train}\PY{p}{,} \PY{n}{y\PYZus{}train}           \PY{o}{=} \PY{n}{mnist}\PY{o}{.}\PY{n}{train}\PY{o}{.}\PY{n}{images}\PY{p}{,} \PY{n}{mnist}\PY{o}{.}\PY{n}{train}\PY{o}{.}\PY{n}{labels}
        \PY{n}{X\PYZus{}validation}\PY{p}{,} \PY{n}{y\PYZus{}validation} \PY{o}{=} \PY{n}{mnist}\PY{o}{.}\PY{n}{validation}\PY{o}{.}\PY{n}{images}\PY{p}{,} \PY{n}{mnist}\PY{o}{.}\PY{n}{validation}\PY{o}{.}\PY{n}{labels}
        \PY{n}{X\PYZus{}test}\PY{p}{,} \PY{n}{y\PYZus{}test}             \PY{o}{=} \PY{n}{mnist}\PY{o}{.}\PY{n}{test}\PY{o}{.}\PY{n}{images}\PY{p}{,} \PY{n}{mnist}\PY{o}{.}\PY{n}{test}\PY{o}{.}\PY{n}{labels}
        
        \PY{k}{assert}\PY{p}{(}\PY{n+nb}{len}\PY{p}{(}\PY{n}{X\PYZus{}train}\PY{p}{)} \PY{o}{==} \PY{n+nb}{len}\PY{p}{(}\PY{n}{y\PYZus{}train}\PY{p}{)}\PY{p}{)}
        \PY{k}{assert}\PY{p}{(}\PY{n+nb}{len}\PY{p}{(}\PY{n}{X\PYZus{}validation}\PY{p}{)} \PY{o}{==} \PY{n+nb}{len}\PY{p}{(}\PY{n}{y\PYZus{}validation}\PY{p}{)}\PY{p}{)}
        \PY{k}{assert}\PY{p}{(}\PY{n+nb}{len}\PY{p}{(}\PY{n}{X\PYZus{}test}\PY{p}{)} \PY{o}{==} \PY{n+nb}{len}\PY{p}{(}\PY{n}{y\PYZus{}test}\PY{p}{)}\PY{p}{)}
        
        \PY{n+nb}{print}\PY{p}{(}\PY{p}{)}
        \PY{n+nb}{print}\PY{p}{(}\PY{l+s+s2}{\PYZdq{}}\PY{l+s+s2}{Image Shape: }\PY{l+s+si}{\PYZob{}\PYZcb{}}\PY{l+s+s2}{\PYZdq{}}\PY{o}{.}\PY{n}{format}\PY{p}{(}\PY{n}{X\PYZus{}train}\PY{p}{[}\PY{l+m+mi}{0}\PY{p}{]}\PY{o}{.}\PY{n}{shape}\PY{p}{)}\PY{p}{)}
        \PY{n+nb}{print}\PY{p}{(}\PY{p}{)}
        \PY{n+nb}{print}\PY{p}{(}\PY{l+s+s2}{\PYZdq{}}\PY{l+s+s2}{Training Set:   }\PY{l+s+si}{\PYZob{}\PYZcb{}}\PY{l+s+s2}{ samples}\PY{l+s+s2}{\PYZdq{}}\PY{o}{.}\PY{n}{format}\PY{p}{(}\PY{n+nb}{len}\PY{p}{(}\PY{n}{X\PYZus{}train}\PY{p}{)}\PY{p}{)}\PY{p}{)}
        \PY{n+nb}{print}\PY{p}{(}\PY{l+s+s2}{\PYZdq{}}\PY{l+s+s2}{Validation Set: }\PY{l+s+si}{\PYZob{}\PYZcb{}}\PY{l+s+s2}{ samples}\PY{l+s+s2}{\PYZdq{}}\PY{o}{.}\PY{n}{format}\PY{p}{(}\PY{n+nb}{len}\PY{p}{(}\PY{n}{X\PYZus{}validation}\PY{p}{)}\PY{p}{)}\PY{p}{)}
        \PY{n+nb}{print}\PY{p}{(}\PY{l+s+s2}{\PYZdq{}}\PY{l+s+s2}{Test Set:       }\PY{l+s+si}{\PYZob{}\PYZcb{}}\PY{l+s+s2}{ samples}\PY{l+s+s2}{\PYZdq{}}\PY{o}{.}\PY{n}{format}\PY{p}{(}\PY{n+nb}{len}\PY{p}{(}\PY{n}{X\PYZus{}test}\PY{p}{)}\PY{p}{)}\PY{p}{)}
\end{Verbatim}


    \begin{Verbatim}[commandchars=\\\{\}]
Extracting MNIST\_data/train-images-idx3-ubyte.gz
Extracting MNIST\_data/train-labels-idx1-ubyte.gz
Extracting MNIST\_data/t10k-images-idx3-ubyte.gz
Extracting MNIST\_data/t10k-labels-idx1-ubyte.gz

Image Shape: (28, 28, 1)

Training Set:   55000 samples
Validation Set: 5000 samples
Test Set:       10000 samples

    \end{Verbatim}

    The MNIST data that TensorFlow pre-loads comes as 28x28x1 images.

However, the LeNet architecture only accepts 32x32xC images, where C is
the number of color channels.

In order to reformat the MNIST data into a shape that LeNet will accept,
we pad the data with two rows of zeros on the top and bottom, and two
columns of zeros on the left and right (28+2+2 = 32).

You do not need to modify this section.

    \begin{Verbatim}[commandchars=\\\{\}]
{\color{incolor}In [{\color{incolor}2}]:} \PY{k+kn}{import} \PY{n+nn}{numpy} \PY{k}{as} \PY{n+nn}{np}
        
        \PY{c+c1}{\PYZsh{} Pad images with 0s}
        \PY{n}{X\PYZus{}train}      \PY{o}{=} \PY{n}{np}\PY{o}{.}\PY{n}{pad}\PY{p}{(}\PY{n}{X\PYZus{}train}\PY{p}{,} \PY{p}{(}\PY{p}{(}\PY{l+m+mi}{0}\PY{p}{,}\PY{l+m+mi}{0}\PY{p}{)}\PY{p}{,}\PY{p}{(}\PY{l+m+mi}{2}\PY{p}{,}\PY{l+m+mi}{2}\PY{p}{)}\PY{p}{,}\PY{p}{(}\PY{l+m+mi}{2}\PY{p}{,}\PY{l+m+mi}{2}\PY{p}{)}\PY{p}{,}\PY{p}{(}\PY{l+m+mi}{0}\PY{p}{,}\PY{l+m+mi}{0}\PY{p}{)}\PY{p}{)}\PY{p}{,} \PY{l+s+s1}{\PYZsq{}}\PY{l+s+s1}{constant}\PY{l+s+s1}{\PYZsq{}}\PY{p}{)}
        \PY{n}{X\PYZus{}validation} \PY{o}{=} \PY{n}{np}\PY{o}{.}\PY{n}{pad}\PY{p}{(}\PY{n}{X\PYZus{}validation}\PY{p}{,} \PY{p}{(}\PY{p}{(}\PY{l+m+mi}{0}\PY{p}{,}\PY{l+m+mi}{0}\PY{p}{)}\PY{p}{,}\PY{p}{(}\PY{l+m+mi}{2}\PY{p}{,}\PY{l+m+mi}{2}\PY{p}{)}\PY{p}{,}\PY{p}{(}\PY{l+m+mi}{2}\PY{p}{,}\PY{l+m+mi}{2}\PY{p}{)}\PY{p}{,}\PY{p}{(}\PY{l+m+mi}{0}\PY{p}{,}\PY{l+m+mi}{0}\PY{p}{)}\PY{p}{)}\PY{p}{,} \PY{l+s+s1}{\PYZsq{}}\PY{l+s+s1}{constant}\PY{l+s+s1}{\PYZsq{}}\PY{p}{)}
        \PY{n}{X\PYZus{}test}       \PY{o}{=} \PY{n}{np}\PY{o}{.}\PY{n}{pad}\PY{p}{(}\PY{n}{X\PYZus{}test}\PY{p}{,} \PY{p}{(}\PY{p}{(}\PY{l+m+mi}{0}\PY{p}{,}\PY{l+m+mi}{0}\PY{p}{)}\PY{p}{,}\PY{p}{(}\PY{l+m+mi}{2}\PY{p}{,}\PY{l+m+mi}{2}\PY{p}{)}\PY{p}{,}\PY{p}{(}\PY{l+m+mi}{2}\PY{p}{,}\PY{l+m+mi}{2}\PY{p}{)}\PY{p}{,}\PY{p}{(}\PY{l+m+mi}{0}\PY{p}{,}\PY{l+m+mi}{0}\PY{p}{)}\PY{p}{)}\PY{p}{,} \PY{l+s+s1}{\PYZsq{}}\PY{l+s+s1}{constant}\PY{l+s+s1}{\PYZsq{}}\PY{p}{)}
            
        \PY{n+nb}{print}\PY{p}{(}\PY{l+s+s2}{\PYZdq{}}\PY{l+s+s2}{Updated Image Shape: }\PY{l+s+si}{\PYZob{}\PYZcb{}}\PY{l+s+s2}{\PYZdq{}}\PY{o}{.}\PY{n}{format}\PY{p}{(}\PY{n}{X\PYZus{}train}\PY{p}{[}\PY{l+m+mi}{0}\PY{p}{]}\PY{o}{.}\PY{n}{shape}\PY{p}{)}\PY{p}{)}
\end{Verbatim}


    \begin{Verbatim}[commandchars=\\\{\}]
Updated Image Shape: (32, 32, 1)

    \end{Verbatim}

    \hypertarget{visualize-data}{%
\subsection{Visualize Data}\label{visualize-data}}

View a sample from the dataset.

You do not need to modify this section.

    \begin{Verbatim}[commandchars=\\\{\}]
{\color{incolor}In [{\color{incolor}10}]:} \PY{k+kn}{import} \PY{n+nn}{random}
         \PY{k+kn}{import} \PY{n+nn}{numpy} \PY{k}{as} \PY{n+nn}{np}
         \PY{k+kn}{import} \PY{n+nn}{matplotlib}\PY{n+nn}{.}\PY{n+nn}{pyplot} \PY{k}{as} \PY{n+nn}{plt}
         \PY{o}{\PYZpc{}}\PY{k}{matplotlib} inline
         
         \PY{n}{index} \PY{o}{=} \PY{n}{random}\PY{o}{.}\PY{n}{randint}\PY{p}{(}\PY{l+m+mi}{0}\PY{p}{,} \PY{n+nb}{len}\PY{p}{(}\PY{n}{X\PYZus{}train}\PY{p}{)}\PY{p}{)}
         \PY{n}{image} \PY{o}{=} \PY{n}{X\PYZus{}train}\PY{p}{[}\PY{n}{index}\PY{p}{]}\PY{o}{.}\PY{n}{squeeze}\PY{p}{(}\PY{p}{)}
         
         \PY{n}{plt}\PY{o}{.}\PY{n}{figure}\PY{p}{(}\PY{n}{figsize}\PY{o}{=}\PY{p}{(}\PY{l+m+mi}{2}\PY{p}{,}\PY{l+m+mi}{2}\PY{p}{)}\PY{p}{)}
         \PY{n}{plt}\PY{o}{.}\PY{n}{imshow}\PY{p}{(}\PY{n}{image}\PY{p}{,} \PY{n}{cmap}\PY{o}{=}\PY{l+s+s2}{\PYZdq{}}\PY{l+s+s2}{gray}\PY{l+s+s2}{\PYZdq{}}\PY{p}{)}
         \PY{n+nb}{print}\PY{p}{(}\PY{n}{y\PYZus{}train}\PY{p}{[}\PY{n}{index}\PY{p}{]}\PY{p}{)}
\end{Verbatim}


    \begin{Verbatim}[commandchars=\\\{\}]
5

    \end{Verbatim}

    \begin{center}
    \adjustimage{max size={0.9\linewidth}{0.9\paperheight}}{output_6_1.png}
    \end{center}
    { \hspace*{\fill} \\}
    
    \hypertarget{preprocess-data}{%
\subsection{Preprocess Data}\label{preprocess-data}}

Shuffle the training data.

You do not need to modify this section.

    \begin{Verbatim}[commandchars=\\\{\}]
{\color{incolor}In [{\color{incolor}11}]:} \PY{k+kn}{from} \PY{n+nn}{sklearn}\PY{n+nn}{.}\PY{n+nn}{utils} \PY{k}{import} \PY{n}{shuffle}
         
         \PY{n}{X\PYZus{}train}\PY{p}{,} \PY{n}{y\PYZus{}train} \PY{o}{=} \PY{n}{shuffle}\PY{p}{(}\PY{n}{X\PYZus{}train}\PY{p}{,} \PY{n}{y\PYZus{}train}\PY{p}{)}
\end{Verbatim}


    \hypertarget{setup-tensorflow}{%
\subsection{Setup TensorFlow}\label{setup-tensorflow}}

The \texttt{EPOCH} and \texttt{BATCH\_SIZE} values affect the training
speed and model accuracy.

You do not need to modify this section.

    \begin{Verbatim}[commandchars=\\\{\}]
{\color{incolor}In [{\color{incolor}12}]:} \PY{k+kn}{import} \PY{n+nn}{tensorflow} \PY{k}{as} \PY{n+nn}{tf}
         
         \PY{n}{EPOCHS} \PY{o}{=} \PY{l+m+mi}{10}
         \PY{n}{BATCH\PYZus{}SIZE} \PY{o}{=} \PY{l+m+mi}{128}
\end{Verbatim}


    \hypertarget{solution-implement-lenet-5}{%
\subsection{SOLUTION: Implement
LeNet-5}\label{solution-implement-lenet-5}}

Implement the \href{http://yann.lecun.com/exdb/lenet/}{LeNet-5} neural
network architecture.

This is the only cell you need to edit. \#\#\# Input The LeNet
architecture accepts a 32x32xC image as input, where C is the number of
color channels. Since MNIST images are grayscale, C is 1 in this case.

\hypertarget{architecture}{%
\subsubsection{Architecture}\label{architecture}}

\textbf{Layer 1: Convolutional.} The output shape should be 28x28x6.

\textbf{Activation.} Your choice of activation function.

\textbf{Pooling.} The output shape should be 14x14x6.

\textbf{Layer 2: Convolutional.} The output shape should be 10x10x16.

\textbf{Activation.} Your choice of activation function.

\textbf{Pooling.} The output shape should be 5x5x16.

\textbf{Flatten.} Flatten the output shape of the final pooling layer
such that it's 1D instead of 3D. The easiest way to do is by using
\texttt{tf.contrib.layers.flatten}, which is already imported for you.

\textbf{Layer 3: Fully Connected.} This should have 120 outputs.

\textbf{Activation.} Your choice of activation function.

\textbf{Layer 4: Fully Connected.} This should have 84 outputs.

\textbf{Activation.} Your choice of activation function.

\textbf{Layer 5: Fully Connected (Logits).} This should have 10 outputs.

\hypertarget{output}{%
\subsubsection{Output}\label{output}}

Return the result of the 2nd fully connected layer.

    \begin{Verbatim}[commandchars=\\\{\}]
{\color{incolor}In [{\color{incolor}13}]:} \PY{k+kn}{from} \PY{n+nn}{tensorflow}\PY{n+nn}{.}\PY{n+nn}{contrib}\PY{n+nn}{.}\PY{n+nn}{layers} \PY{k}{import} \PY{n}{flatten}
         
         \PY{k}{def} \PY{n+nf}{LeNet}\PY{p}{(}\PY{n}{x}\PY{p}{)}\PY{p}{:}    
             \PY{c+c1}{\PYZsh{} Arguments used for tf.truncated\PYZus{}normal, randomly defines variables for the weights and biases for each layer}
             \PY{n}{mu} \PY{o}{=} \PY{l+m+mi}{0}
             \PY{n}{sigma} \PY{o}{=} \PY{l+m+mf}{0.1}
             
             \PY{c+c1}{\PYZsh{} SOLUTION: Layer 1: Convolutional. Input = 32x32x1. Output = 28x28x6.}
             \PY{n}{conv1\PYZus{}W} \PY{o}{=} \PY{n}{tf}\PY{o}{.}\PY{n}{Variable}\PY{p}{(}\PY{n}{tf}\PY{o}{.}\PY{n}{truncated\PYZus{}normal}\PY{p}{(}\PY{n}{shape}\PY{o}{=}\PY{p}{(}\PY{l+m+mi}{5}\PY{p}{,} \PY{l+m+mi}{5}\PY{p}{,} \PY{l+m+mi}{1}\PY{p}{,} \PY{l+m+mi}{6}\PY{p}{)}\PY{p}{,} \PY{n}{mean} \PY{o}{=} \PY{n}{mu}\PY{p}{,} \PY{n}{stddev} \PY{o}{=} \PY{n}{sigma}\PY{p}{)}\PY{p}{)}
             \PY{n}{conv1\PYZus{}b} \PY{o}{=} \PY{n}{tf}\PY{o}{.}\PY{n}{Variable}\PY{p}{(}\PY{n}{tf}\PY{o}{.}\PY{n}{zeros}\PY{p}{(}\PY{l+m+mi}{6}\PY{p}{)}\PY{p}{)}
             \PY{n}{conv1}   \PY{o}{=} \PY{n}{tf}\PY{o}{.}\PY{n}{nn}\PY{o}{.}\PY{n}{conv2d}\PY{p}{(}\PY{n}{x}\PY{p}{,} \PY{n}{conv1\PYZus{}W}\PY{p}{,} \PY{n}{strides}\PY{o}{=}\PY{p}{[}\PY{l+m+mi}{1}\PY{p}{,} \PY{l+m+mi}{1}\PY{p}{,} \PY{l+m+mi}{1}\PY{p}{,} \PY{l+m+mi}{1}\PY{p}{]}\PY{p}{,} \PY{n}{padding}\PY{o}{=}\PY{l+s+s1}{\PYZsq{}}\PY{l+s+s1}{VALID}\PY{l+s+s1}{\PYZsq{}}\PY{p}{)} \PY{o}{+} \PY{n}{conv1\PYZus{}b}
         
             \PY{c+c1}{\PYZsh{} SOLUTION: Activation.}
             \PY{n}{conv1} \PY{o}{=} \PY{n}{tf}\PY{o}{.}\PY{n}{nn}\PY{o}{.}\PY{n}{relu}\PY{p}{(}\PY{n}{conv1}\PY{p}{)}
         
             \PY{c+c1}{\PYZsh{} SOLUTION: Pooling. Input = 28x28x6. Output = 14x14x6.}
             \PY{n}{conv1} \PY{o}{=} \PY{n}{tf}\PY{o}{.}\PY{n}{nn}\PY{o}{.}\PY{n}{max\PYZus{}pool}\PY{p}{(}\PY{n}{conv1}\PY{p}{,} \PY{n}{ksize}\PY{o}{=}\PY{p}{[}\PY{l+m+mi}{1}\PY{p}{,} \PY{l+m+mi}{2}\PY{p}{,} \PY{l+m+mi}{2}\PY{p}{,} \PY{l+m+mi}{1}\PY{p}{]}\PY{p}{,} \PY{n}{strides}\PY{o}{=}\PY{p}{[}\PY{l+m+mi}{1}\PY{p}{,} \PY{l+m+mi}{2}\PY{p}{,} \PY{l+m+mi}{2}\PY{p}{,} \PY{l+m+mi}{1}\PY{p}{]}\PY{p}{,} \PY{n}{padding}\PY{o}{=}\PY{l+s+s1}{\PYZsq{}}\PY{l+s+s1}{VALID}\PY{l+s+s1}{\PYZsq{}}\PY{p}{)}
         
             \PY{c+c1}{\PYZsh{} SOLUTION: Layer 2: Convolutional. Output = 10x10x16.}
             \PY{n}{conv2\PYZus{}W} \PY{o}{=} \PY{n}{tf}\PY{o}{.}\PY{n}{Variable}\PY{p}{(}\PY{n}{tf}\PY{o}{.}\PY{n}{truncated\PYZus{}normal}\PY{p}{(}\PY{n}{shape}\PY{o}{=}\PY{p}{(}\PY{l+m+mi}{5}\PY{p}{,} \PY{l+m+mi}{5}\PY{p}{,} \PY{l+m+mi}{6}\PY{p}{,} \PY{l+m+mi}{16}\PY{p}{)}\PY{p}{,} \PY{n}{mean} \PY{o}{=} \PY{n}{mu}\PY{p}{,} \PY{n}{stddev} \PY{o}{=} \PY{n}{sigma}\PY{p}{)}\PY{p}{)}
             \PY{n}{conv2\PYZus{}b} \PY{o}{=} \PY{n}{tf}\PY{o}{.}\PY{n}{Variable}\PY{p}{(}\PY{n}{tf}\PY{o}{.}\PY{n}{zeros}\PY{p}{(}\PY{l+m+mi}{16}\PY{p}{)}\PY{p}{)}
             \PY{n}{conv2}   \PY{o}{=} \PY{n}{tf}\PY{o}{.}\PY{n}{nn}\PY{o}{.}\PY{n}{conv2d}\PY{p}{(}\PY{n}{conv1}\PY{p}{,} \PY{n}{conv2\PYZus{}W}\PY{p}{,} \PY{n}{strides}\PY{o}{=}\PY{p}{[}\PY{l+m+mi}{1}\PY{p}{,} \PY{l+m+mi}{1}\PY{p}{,} \PY{l+m+mi}{1}\PY{p}{,} \PY{l+m+mi}{1}\PY{p}{]}\PY{p}{,} \PY{n}{padding}\PY{o}{=}\PY{l+s+s1}{\PYZsq{}}\PY{l+s+s1}{VALID}\PY{l+s+s1}{\PYZsq{}}\PY{p}{)} \PY{o}{+} \PY{n}{conv2\PYZus{}b}
             
             \PY{c+c1}{\PYZsh{} SOLUTION: Activation.}
             \PY{n}{conv2} \PY{o}{=} \PY{n}{tf}\PY{o}{.}\PY{n}{nn}\PY{o}{.}\PY{n}{relu}\PY{p}{(}\PY{n}{conv2}\PY{p}{)}
         
             \PY{c+c1}{\PYZsh{} SOLUTION: Pooling. Input = 10x10x16. Output = 5x5x16.}
             \PY{n}{conv2} \PY{o}{=} \PY{n}{tf}\PY{o}{.}\PY{n}{nn}\PY{o}{.}\PY{n}{max\PYZus{}pool}\PY{p}{(}\PY{n}{conv2}\PY{p}{,} \PY{n}{ksize}\PY{o}{=}\PY{p}{[}\PY{l+m+mi}{1}\PY{p}{,} \PY{l+m+mi}{2}\PY{p}{,} \PY{l+m+mi}{2}\PY{p}{,} \PY{l+m+mi}{1}\PY{p}{]}\PY{p}{,} \PY{n}{strides}\PY{o}{=}\PY{p}{[}\PY{l+m+mi}{1}\PY{p}{,} \PY{l+m+mi}{2}\PY{p}{,} \PY{l+m+mi}{2}\PY{p}{,} \PY{l+m+mi}{1}\PY{p}{]}\PY{p}{,} \PY{n}{padding}\PY{o}{=}\PY{l+s+s1}{\PYZsq{}}\PY{l+s+s1}{VALID}\PY{l+s+s1}{\PYZsq{}}\PY{p}{)}
         
             \PY{c+c1}{\PYZsh{} SOLUTION: Flatten. Input = 5x5x16. Output = 400.}
             \PY{n}{fc0}   \PY{o}{=} \PY{n}{flatten}\PY{p}{(}\PY{n}{conv2}\PY{p}{)}
             
             \PY{c+c1}{\PYZsh{} SOLUTION: Layer 3: Fully Connected. Input = 400. Output = 120.}
             \PY{n}{fc1\PYZus{}W} \PY{o}{=} \PY{n}{tf}\PY{o}{.}\PY{n}{Variable}\PY{p}{(}\PY{n}{tf}\PY{o}{.}\PY{n}{truncated\PYZus{}normal}\PY{p}{(}\PY{n}{shape}\PY{o}{=}\PY{p}{(}\PY{l+m+mi}{400}\PY{p}{,} \PY{l+m+mi}{120}\PY{p}{)}\PY{p}{,} \PY{n}{mean} \PY{o}{=} \PY{n}{mu}\PY{p}{,} \PY{n}{stddev} \PY{o}{=} \PY{n}{sigma}\PY{p}{)}\PY{p}{)}
             \PY{n}{fc1\PYZus{}b} \PY{o}{=} \PY{n}{tf}\PY{o}{.}\PY{n}{Variable}\PY{p}{(}\PY{n}{tf}\PY{o}{.}\PY{n}{zeros}\PY{p}{(}\PY{l+m+mi}{120}\PY{p}{)}\PY{p}{)}
             \PY{n}{fc1}   \PY{o}{=} \PY{n}{tf}\PY{o}{.}\PY{n}{matmul}\PY{p}{(}\PY{n}{fc0}\PY{p}{,} \PY{n}{fc1\PYZus{}W}\PY{p}{)} \PY{o}{+} \PY{n}{fc1\PYZus{}b}
             
             \PY{c+c1}{\PYZsh{} SOLUTION: Activation.}
             \PY{n}{fc1}    \PY{o}{=} \PY{n}{tf}\PY{o}{.}\PY{n}{nn}\PY{o}{.}\PY{n}{relu}\PY{p}{(}\PY{n}{fc1}\PY{p}{)}
         
             \PY{c+c1}{\PYZsh{} SOLUTION: Layer 4: Fully Connected. Input = 120. Output = 84.}
             \PY{n}{fc2\PYZus{}W}  \PY{o}{=} \PY{n}{tf}\PY{o}{.}\PY{n}{Variable}\PY{p}{(}\PY{n}{tf}\PY{o}{.}\PY{n}{truncated\PYZus{}normal}\PY{p}{(}\PY{n}{shape}\PY{o}{=}\PY{p}{(}\PY{l+m+mi}{120}\PY{p}{,} \PY{l+m+mi}{84}\PY{p}{)}\PY{p}{,} \PY{n}{mean} \PY{o}{=} \PY{n}{mu}\PY{p}{,} \PY{n}{stddev} \PY{o}{=} \PY{n}{sigma}\PY{p}{)}\PY{p}{)}
             \PY{n}{fc2\PYZus{}b}  \PY{o}{=} \PY{n}{tf}\PY{o}{.}\PY{n}{Variable}\PY{p}{(}\PY{n}{tf}\PY{o}{.}\PY{n}{zeros}\PY{p}{(}\PY{l+m+mi}{84}\PY{p}{)}\PY{p}{)}
             \PY{n}{fc2}    \PY{o}{=} \PY{n}{tf}\PY{o}{.}\PY{n}{matmul}\PY{p}{(}\PY{n}{fc1}\PY{p}{,} \PY{n}{fc2\PYZus{}W}\PY{p}{)} \PY{o}{+} \PY{n}{fc2\PYZus{}b}
             
             \PY{c+c1}{\PYZsh{} SOLUTION: Activation.}
             \PY{n}{fc2}    \PY{o}{=} \PY{n}{tf}\PY{o}{.}\PY{n}{nn}\PY{o}{.}\PY{n}{relu}\PY{p}{(}\PY{n}{fc2}\PY{p}{)}
         
             \PY{c+c1}{\PYZsh{} SOLUTION: Layer 5: Fully Connected. Input = 84. Output = 10.}
             \PY{n}{fc3\PYZus{}W}  \PY{o}{=} \PY{n}{tf}\PY{o}{.}\PY{n}{Variable}\PY{p}{(}\PY{n}{tf}\PY{o}{.}\PY{n}{truncated\PYZus{}normal}\PY{p}{(}\PY{n}{shape}\PY{o}{=}\PY{p}{(}\PY{l+m+mi}{84}\PY{p}{,} \PY{l+m+mi}{10}\PY{p}{)}\PY{p}{,} \PY{n}{mean} \PY{o}{=} \PY{n}{mu}\PY{p}{,} \PY{n}{stddev} \PY{o}{=} \PY{n}{sigma}\PY{p}{)}\PY{p}{)}
             \PY{n}{fc3\PYZus{}b}  \PY{o}{=} \PY{n}{tf}\PY{o}{.}\PY{n}{Variable}\PY{p}{(}\PY{n}{tf}\PY{o}{.}\PY{n}{zeros}\PY{p}{(}\PY{l+m+mi}{10}\PY{p}{)}\PY{p}{)}
             \PY{n}{logits} \PY{o}{=} \PY{n}{tf}\PY{o}{.}\PY{n}{matmul}\PY{p}{(}\PY{n}{fc2}\PY{p}{,} \PY{n}{fc3\PYZus{}W}\PY{p}{)} \PY{o}{+} \PY{n}{fc3\PYZus{}b}
             
             \PY{k}{return} \PY{n}{logits}
\end{Verbatim}


    \hypertarget{features-and-labels}{%
\subsection{Features and Labels}\label{features-and-labels}}

Train LeNet to classify \href{http://yann.lecun.com/exdb/mnist/}{MNIST}
data.

\texttt{x} is a placeholder for a batch of input images. \texttt{y} is a
placeholder for a batch of output labels.

You do not need to modify this section.

    \begin{Verbatim}[commandchars=\\\{\}]
{\color{incolor}In [{\color{incolor}14}]:} \PY{n}{x} \PY{o}{=} \PY{n}{tf}\PY{o}{.}\PY{n}{placeholder}\PY{p}{(}\PY{n}{tf}\PY{o}{.}\PY{n}{float32}\PY{p}{,} \PY{p}{(}\PY{k+kc}{None}\PY{p}{,} \PY{l+m+mi}{32}\PY{p}{,} \PY{l+m+mi}{32}\PY{p}{,} \PY{l+m+mi}{1}\PY{p}{)}\PY{p}{)}
         \PY{n}{y} \PY{o}{=} \PY{n}{tf}\PY{o}{.}\PY{n}{placeholder}\PY{p}{(}\PY{n}{tf}\PY{o}{.}\PY{n}{int32}\PY{p}{,} \PY{p}{(}\PY{k+kc}{None}\PY{p}{)}\PY{p}{)}
         \PY{n}{one\PYZus{}hot\PYZus{}y} \PY{o}{=} \PY{n}{tf}\PY{o}{.}\PY{n}{one\PYZus{}hot}\PY{p}{(}\PY{n}{y}\PY{p}{,} \PY{l+m+mi}{10}\PY{p}{)}
\end{Verbatim}


    \hypertarget{training-pipeline}{%
\subsection{Training Pipeline}\label{training-pipeline}}

Create a training pipeline that uses the model to classify MNIST data.

You do not need to modify this section.

    \begin{Verbatim}[commandchars=\\\{\}]
{\color{incolor}In [{\color{incolor}15}]:} \PY{n}{rate} \PY{o}{=} \PY{l+m+mf}{0.001}
         
         \PY{n}{logits} \PY{o}{=} \PY{n}{LeNet}\PY{p}{(}\PY{n}{x}\PY{p}{)}
         \PY{n}{cross\PYZus{}entropy} \PY{o}{=} \PY{n}{tf}\PY{o}{.}\PY{n}{nn}\PY{o}{.}\PY{n}{softmax\PYZus{}cross\PYZus{}entropy\PYZus{}with\PYZus{}logits}\PY{p}{(}\PY{n}{labels}\PY{o}{=}\PY{n}{one\PYZus{}hot\PYZus{}y}\PY{p}{,} \PY{n}{logits}\PY{o}{=}\PY{n}{logits}\PY{p}{)}
         \PY{n}{loss\PYZus{}operation} \PY{o}{=} \PY{n}{tf}\PY{o}{.}\PY{n}{reduce\PYZus{}mean}\PY{p}{(}\PY{n}{cross\PYZus{}entropy}\PY{p}{)}
         \PY{n}{optimizer} \PY{o}{=} \PY{n}{tf}\PY{o}{.}\PY{n}{train}\PY{o}{.}\PY{n}{AdamOptimizer}\PY{p}{(}\PY{n}{learning\PYZus{}rate} \PY{o}{=} \PY{n}{rate}\PY{p}{)}
         \PY{n}{training\PYZus{}operation} \PY{o}{=} \PY{n}{optimizer}\PY{o}{.}\PY{n}{minimize}\PY{p}{(}\PY{n}{loss\PYZus{}operation}\PY{p}{)}
\end{Verbatim}


    \hypertarget{model-evaluation}{%
\subsection{Model Evaluation}\label{model-evaluation}}

Evaluate how well the loss and accuracy of the model for a given
dataset.

You do not need to modify this section.

    \begin{Verbatim}[commandchars=\\\{\}]
{\color{incolor}In [{\color{incolor}16}]:} \PY{n}{correct\PYZus{}prediction} \PY{o}{=} \PY{n}{tf}\PY{o}{.}\PY{n}{equal}\PY{p}{(}\PY{n}{tf}\PY{o}{.}\PY{n}{argmax}\PY{p}{(}\PY{n}{logits}\PY{p}{,} \PY{l+m+mi}{1}\PY{p}{)}\PY{p}{,} \PY{n}{tf}\PY{o}{.}\PY{n}{argmax}\PY{p}{(}\PY{n}{one\PYZus{}hot\PYZus{}y}\PY{p}{,} \PY{l+m+mi}{1}\PY{p}{)}\PY{p}{)}
         \PY{n}{accuracy\PYZus{}operation} \PY{o}{=} \PY{n}{tf}\PY{o}{.}\PY{n}{reduce\PYZus{}mean}\PY{p}{(}\PY{n}{tf}\PY{o}{.}\PY{n}{cast}\PY{p}{(}\PY{n}{correct\PYZus{}prediction}\PY{p}{,} \PY{n}{tf}\PY{o}{.}\PY{n}{float32}\PY{p}{)}\PY{p}{)}
         \PY{n}{saver} \PY{o}{=} \PY{n}{tf}\PY{o}{.}\PY{n}{train}\PY{o}{.}\PY{n}{Saver}\PY{p}{(}\PY{p}{)}
         
         \PY{k}{def} \PY{n+nf}{evaluate}\PY{p}{(}\PY{n}{X\PYZus{}data}\PY{p}{,} \PY{n}{y\PYZus{}data}\PY{p}{)}\PY{p}{:}
             \PY{n}{num\PYZus{}examples} \PY{o}{=} \PY{n+nb}{len}\PY{p}{(}\PY{n}{X\PYZus{}data}\PY{p}{)}
             \PY{n}{total\PYZus{}accuracy} \PY{o}{=} \PY{l+m+mi}{0}
             \PY{n}{sess} \PY{o}{=} \PY{n}{tf}\PY{o}{.}\PY{n}{get\PYZus{}default\PYZus{}session}\PY{p}{(}\PY{p}{)}
             \PY{k}{for} \PY{n}{offset} \PY{o+ow}{in} \PY{n+nb}{range}\PY{p}{(}\PY{l+m+mi}{0}\PY{p}{,} \PY{n}{num\PYZus{}examples}\PY{p}{,} \PY{n}{BATCH\PYZus{}SIZE}\PY{p}{)}\PY{p}{:}
                 \PY{n}{batch\PYZus{}x}\PY{p}{,} \PY{n}{batch\PYZus{}y} \PY{o}{=} \PY{n}{X\PYZus{}data}\PY{p}{[}\PY{n}{offset}\PY{p}{:}\PY{n}{offset}\PY{o}{+}\PY{n}{BATCH\PYZus{}SIZE}\PY{p}{]}\PY{p}{,} \PY{n}{y\PYZus{}data}\PY{p}{[}\PY{n}{offset}\PY{p}{:}\PY{n}{offset}\PY{o}{+}\PY{n}{BATCH\PYZus{}SIZE}\PY{p}{]}
                 \PY{n}{accuracy} \PY{o}{=} \PY{n}{sess}\PY{o}{.}\PY{n}{run}\PY{p}{(}\PY{n}{accuracy\PYZus{}operation}\PY{p}{,} \PY{n}{feed\PYZus{}dict}\PY{o}{=}\PY{p}{\PYZob{}}\PY{n}{x}\PY{p}{:} \PY{n}{batch\PYZus{}x}\PY{p}{,} \PY{n}{y}\PY{p}{:} \PY{n}{batch\PYZus{}y}\PY{p}{\PYZcb{}}\PY{p}{)}
                 \PY{n}{total\PYZus{}accuracy} \PY{o}{+}\PY{o}{=} \PY{p}{(}\PY{n}{accuracy} \PY{o}{*} \PY{n+nb}{len}\PY{p}{(}\PY{n}{batch\PYZus{}x}\PY{p}{)}\PY{p}{)}
             \PY{k}{return} \PY{n}{total\PYZus{}accuracy} \PY{o}{/} \PY{n}{num\PYZus{}examples}
\end{Verbatim}


    \hypertarget{train-the-model}{%
\subsection{Train the Model}\label{train-the-model}}

Run the training data through the training pipeline to train the model.

Before each epoch, shuffle the training set.

After each epoch, measure the loss and accuracy of the validation set.

Save the model after training.

You do not need to modify this section.

    \begin{Verbatim}[commandchars=\\\{\}]
{\color{incolor}In [{\color{incolor}17}]:} \PY{k}{with} \PY{n}{tf}\PY{o}{.}\PY{n}{Session}\PY{p}{(}\PY{p}{)} \PY{k}{as} \PY{n}{sess}\PY{p}{:}
             \PY{n}{sess}\PY{o}{.}\PY{n}{run}\PY{p}{(}\PY{n}{tf}\PY{o}{.}\PY{n}{global\PYZus{}variables\PYZus{}initializer}\PY{p}{(}\PY{p}{)}\PY{p}{)}
             \PY{n}{num\PYZus{}examples} \PY{o}{=} \PY{n+nb}{len}\PY{p}{(}\PY{n}{X\PYZus{}train}\PY{p}{)}
             
             \PY{n+nb}{print}\PY{p}{(}\PY{l+s+s2}{\PYZdq{}}\PY{l+s+s2}{Training...}\PY{l+s+s2}{\PYZdq{}}\PY{p}{)}
             \PY{n+nb}{print}\PY{p}{(}\PY{p}{)}
             \PY{k}{for} \PY{n}{i} \PY{o+ow}{in} \PY{n+nb}{range}\PY{p}{(}\PY{n}{EPOCHS}\PY{p}{)}\PY{p}{:}
                 \PY{n}{X\PYZus{}train}\PY{p}{,} \PY{n}{y\PYZus{}train} \PY{o}{=} \PY{n}{shuffle}\PY{p}{(}\PY{n}{X\PYZus{}train}\PY{p}{,} \PY{n}{y\PYZus{}train}\PY{p}{)}
                 \PY{k}{for} \PY{n}{offset} \PY{o+ow}{in} \PY{n+nb}{range}\PY{p}{(}\PY{l+m+mi}{0}\PY{p}{,} \PY{n}{num\PYZus{}examples}\PY{p}{,} \PY{n}{BATCH\PYZus{}SIZE}\PY{p}{)}\PY{p}{:}
                     \PY{n}{end} \PY{o}{=} \PY{n}{offset} \PY{o}{+} \PY{n}{BATCH\PYZus{}SIZE}
                     \PY{n}{batch\PYZus{}x}\PY{p}{,} \PY{n}{batch\PYZus{}y} \PY{o}{=} \PY{n}{X\PYZus{}train}\PY{p}{[}\PY{n}{offset}\PY{p}{:}\PY{n}{end}\PY{p}{]}\PY{p}{,} \PY{n}{y\PYZus{}train}\PY{p}{[}\PY{n}{offset}\PY{p}{:}\PY{n}{end}\PY{p}{]}
                     \PY{n}{sess}\PY{o}{.}\PY{n}{run}\PY{p}{(}\PY{n}{training\PYZus{}operation}\PY{p}{,} \PY{n}{feed\PYZus{}dict}\PY{o}{=}\PY{p}{\PYZob{}}\PY{n}{x}\PY{p}{:} \PY{n}{batch\PYZus{}x}\PY{p}{,} \PY{n}{y}\PY{p}{:} \PY{n}{batch\PYZus{}y}\PY{p}{\PYZcb{}}\PY{p}{)}
                     
                 \PY{n}{validation\PYZus{}accuracy} \PY{o}{=} \PY{n}{evaluate}\PY{p}{(}\PY{n}{X\PYZus{}validation}\PY{p}{,} \PY{n}{y\PYZus{}validation}\PY{p}{)}
                 \PY{n+nb}{print}\PY{p}{(}\PY{l+s+s2}{\PYZdq{}}\PY{l+s+s2}{EPOCH }\PY{l+s+si}{\PYZob{}\PYZcb{}}\PY{l+s+s2}{ ...}\PY{l+s+s2}{\PYZdq{}}\PY{o}{.}\PY{n}{format}\PY{p}{(}\PY{n}{i}\PY{o}{+}\PY{l+m+mi}{1}\PY{p}{)}\PY{p}{)}
                 \PY{n+nb}{print}\PY{p}{(}\PY{l+s+s2}{\PYZdq{}}\PY{l+s+s2}{Validation Accuracy = }\PY{l+s+si}{\PYZob{}:.3f\PYZcb{}}\PY{l+s+s2}{\PYZdq{}}\PY{o}{.}\PY{n}{format}\PY{p}{(}\PY{n}{validation\PYZus{}accuracy}\PY{p}{)}\PY{p}{)}
                 \PY{n+nb}{print}\PY{p}{(}\PY{p}{)}
                 
             \PY{n}{saver}\PY{o}{.}\PY{n}{save}\PY{p}{(}\PY{n}{sess}\PY{p}{,} \PY{l+s+s1}{\PYZsq{}}\PY{l+s+s1}{./lenet}\PY{l+s+s1}{\PYZsq{}}\PY{p}{)}
             \PY{n+nb}{print}\PY{p}{(}\PY{l+s+s2}{\PYZdq{}}\PY{l+s+s2}{Model saved}\PY{l+s+s2}{\PYZdq{}}\PY{p}{)}
\end{Verbatim}


    \begin{Verbatim}[commandchars=\\\{\}]
Training{\ldots}

EPOCH 1 {\ldots}
Validation Accuracy = 0.969

EPOCH 2 {\ldots}
Validation Accuracy = 0.981

EPOCH 3 {\ldots}
Validation Accuracy = 0.984

EPOCH 4 {\ldots}
Validation Accuracy = 0.984

EPOCH 5 {\ldots}
Validation Accuracy = 0.987

EPOCH 6 {\ldots}
Validation Accuracy = 0.987

EPOCH 7 {\ldots}
Validation Accuracy = 0.987

EPOCH 8 {\ldots}
Validation Accuracy = 0.990

EPOCH 9 {\ldots}
Validation Accuracy = 0.990

EPOCH 10 {\ldots}
Validation Accuracy = 0.990

Model saved

    \end{Verbatim}

    \hypertarget{evaluate-the-model}{%
\subsection{Evaluate the Model}\label{evaluate-the-model}}

Once you are completely satisfied with your model, evaluate the
performance of the model on the test set.

Be sure to only do this once!

If you were to measure the performance of your trained model on the test
set, then improve your model, and then measure the performance of your
model on the test set again, that would invalidate your test results.
You wouldn't get a true measure of how well your model would perform
against real data.

You do not need to modify this section.

    \begin{Verbatim}[commandchars=\\\{\}]
{\color{incolor}In [{\color{incolor}18}]:} \PY{k}{with} \PY{n}{tf}\PY{o}{.}\PY{n}{Session}\PY{p}{(}\PY{p}{)} \PY{k}{as} \PY{n}{sess}\PY{p}{:}
             \PY{n}{saver}\PY{o}{.}\PY{n}{restore}\PY{p}{(}\PY{n}{sess}\PY{p}{,} \PY{n}{tf}\PY{o}{.}\PY{n}{train}\PY{o}{.}\PY{n}{latest\PYZus{}checkpoint}\PY{p}{(}\PY{l+s+s1}{\PYZsq{}}\PY{l+s+s1}{.}\PY{l+s+s1}{\PYZsq{}}\PY{p}{)}\PY{p}{)}
         
             \PY{n}{test\PYZus{}accuracy} \PY{o}{=} \PY{n}{evaluate}\PY{p}{(}\PY{n}{X\PYZus{}test}\PY{p}{,} \PY{n}{y\PYZus{}test}\PY{p}{)}
             \PY{n+nb}{print}\PY{p}{(}\PY{l+s+s2}{\PYZdq{}}\PY{l+s+s2}{Test Accuracy = }\PY{l+s+si}{\PYZob{}:.3f\PYZcb{}}\PY{l+s+s2}{\PYZdq{}}\PY{o}{.}\PY{n}{format}\PY{p}{(}\PY{n}{test\PYZus{}accuracy}\PY{p}{)}\PY{p}{)}
\end{Verbatim}


    \begin{Verbatim}[commandchars=\\\{\}]
INFO:tensorflow:Restoring parameters from ./lenet
Test Accuracy = 0.989

    \end{Verbatim}


    % Add a bibliography block to the postdoc
    
    
    
    \end{document}
